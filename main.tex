\documentclass[a4paper,12pt]{article}
\usepackage[utf8]{inputenc}
\usepackage[T2A]{fontenc}
\usepackage[russian]{babel}

\usepackage{float}

% margins
\usepackage[a4paper, left=30mm,
 top=30mm,  right=30mm,
 bottom=20mm]{geometry}

\usepackage{amsmath}
\usepackage{amssymb}
\usepackage{amsthm}


% correct inequalities symbols
\let\leq\leqslant
\let\geq\geqslant


\newenvironment{answer}{\par\emph{Ответ:}}{\par}

\usepackage{csquotes}

\usepackage{mathrsfs}

\usepackage{indentfirst}

\usepackage{color}   %Colored links
\newtheorem{statement}{Утверждение}[section]


\usepackage{hyperref}

\hypersetup{
    colorlinks=true,
    linktoc=all,
    linkcolor=blue,
}

\usepackage[dvipsnames]{xcolor}

\setlength {\marginparwidth }{2cm}
\usepackage{todonotes}

\usepackage[font=small,labelfont=bf, labelformat=empty]{caption}

\usepackage{appendix}

\usepackage{tikz-cd}

\DeclareMathOperator{\Hom}{Hom}

\renewcommand\appendixtocname{Приложения}
\renewcommand\appendixpagename{Приложения}

\begin{document}

\author{Глеб Красилич}
\title{Современная верстка математических текстов в \LaTeX}
\date{Март 2023}
\maketitle

\tableofcontents

\addcontentsline{toc}{section}{Введение}

\section*{Введение}
Система \LaTeX -- незаменимый инструмент для любого академика,
работающего в области математики и теоретической информатики.
Будучи свободно распространяем пакетом, \LaTeX позволяет быстро, удобно и бесплатно
оформлять математические тексты и презентации, а богатая экосистема
пакетов и инструментов позволяет добиться большой гибкости в оформлении
работ. Давайте продемонстрируем некоторые возможности \LaTeX.

\section{Глава первая}

Начнем с самого базового.

\subsection{Ссылки и библиография}

Пакет \LaTeX содержит встроенные инструменты для работы с библиографией
и ссылками на литературу. Так, необходимость написания этого текста
на втором курсе магистратуры заставляет меня вспомнить о \cite{process}
и даже \cite{lovecraft}.

\subsection{Работа с утверждениями}

В любом математическом тексте нам нужно аккуратно оформлять утверждения
и доказательства к ним. \LaTeX позволяет делать это с легкостью:

\begin{statement}[Первая теорема Геделя о неполноте]
Пусть $F$ -- теория первого порядка, удовлетворяющая следующим
условиям:
\begin{enumerate}
    \item $F$ не является противоречивой ($F$ не может доказать константу $\bot$) 
    \item $F$ содержит Арифметику Робинсона $Q$ как подтеорию.
\end{enumerate}
Тогда в языке $F$ существует формула $\phi$ такая, что ни она сама,
ни ее отрицание не выводится в $F$.
\end{statement}
\begin{proof}
    Оставляем как простое упражнение для читателя.
\end{proof}

\subsection{Форматирование}

Хорошо оформленный текст -- залог успеха публикации! \LaTeX дает
нам богатые возможности для форматирования. Например, мы \textcolor{green}{можем} сделать
текст \textcolor{red}{цветным}. Или сделать сноску на \todo[linecolor=green,bordercolor=green,backgroundcolor=green]{ВО!}полях.

Или даже две \todo[linecolor=blue,bordercolor=yellow,backgroundcolor=red]{Во! (x2)}сноски!

\todo[inline, backgroundcolor=yellow, inlinewidth=9.9cm]{Можем делать замечания прямо по среди текста.}

Можем добавлять \href{https://math.hse.ru}{ссылки}!

Можем делать сноски\footnote[1]{Сноска!}, и удобно\footnote[2]{Ну вы поняли} работать с ними\footnotemark[2]! 

\subsection{Работа с иллюстрациями}

Естественно \LaTeX позволяет работать с файлами изображений.
Вот например чудесные картинки, сгенерированный алгоритмом
Midjourney:

\begin{minipage}[t][][t]{0.2\linewidth}
\vspace{3ex}
\begin{center}
\includegraphics[width=71mm]{img/GlebChili_A_sad_cat_trying_to_write_mathematical_text_in_LaTeX._8451f32e-91ec-4e9b-a756-9101632af4a9.png}
\captionof{figure}{A sad cat writing mathematical text in LaTeX.}\addtocounter{figure}{-1}
\end{center}
\end{minipage}\hfill
\begin{minipage}[t][][t]{0.2\linewidth}
\vspace{3ex}
\begin{center}
\includegraphics[width=71mm]{img/GlebChili_Math_professor_checking_emails_from_students._In_styl_e02f6eff-3592-4212-9325-83b12cee4755.png}
\captionof{figure}{Math professor checking emails from students. In style of Pixar.}\addtocounter{figure}{-1}
\end{center}
\end{minipage}\hfill
\captionsetup{labelformat=default}
\captionof{figure}{Работы алгоритма Midjourney}

\section{Продвинутое математическое форматирование}

\subsection{Шрифты}

Существует традиция записывать имена определенных
математических объектов, использую определенные
устоявшиеся шрифты. \LaTeX (c дополнительными пакетами)
позволяет это делать. Пример: $\mathbb{Z}$, $\mathfrak{g}$ и $\mathcal{O}$.

\subsection{Оформление формул}

Вот так мы можем оформлять формулы

\begin{align}
    \label{firstform}
    (X_n, X_m) = \int\limits_0^l \bar{X_n}(x) X_m(x) dx = \lambda^{-1}_n \int \limits_0^l \bar{X_n}(x) \int \limits_0^l G(x, y) X_m(x) dx dy = \\
    = \lambda_n^{-1}(X_m, X_n)(\bar{a_m}, X_m)
\end{align}

\begin{equation}
    \label{secondform}
    \Hom(F(A), B) \cong \Hom(A, G(B))
\end{equation}

Более того, на эти формулы мы можем удобно ссылаться (\ref{firstform}) (\ref{secondform}).

\subsection{Продвинутое математическое форматирование}

Иногда нам нужно продвинутое форматирование. \LaTeX
помогает нам и здесь:

$$
\underbrace{a + \dots a}_{k \text{ раз}} + \overbrace{b + \dots + b}^{l \text{ раз}}
$$

$$
\widetilde{x_N} = \sum_{\substack{1 \leq p \leq N \\ 1 \leq q \leq N \\ | p^2 + q^2 | \leq N}} (\cos px + \sin qx)
$$

\subsection{Коммутативные диаграммы}

Пакет tikz-cd позволяет нам рисовать вот такие
красивые коммутативные диаграммы:

$$
\begin{tikzcd}
& \arrow[bend right=60, swap]{ddl}{\psi_1} P \arrow{d}{s} \arrow[bend left=60]{ddr}{\psi_2}& \\
& \arrow[swap]{ld}{p_1 \circ eq} M \arrow{d}{u} \arrow{rd}{p_2 \circ eq}& \\
A \arrow[swap]{r}{f} & X & \arrow{l}{g} B
\end{tikzcd}
$$

\appendixpage \addappheadtotoc

\begin{appendices}

\section{Приложение I}

Это приложение.

\section{Приложение II}

Это тоже приложение. Кстати, у нас есть знаки
неравенства: $\leq$, $\geq$.

\end{appendices}

\renewcommand\refname{Библиография}
\addcontentsline{toc}{section}{Библиография}

\begin{thebibliography}{}

\bibitem{process} Ф. Кафка. \textit{Процесс.} ФТМ, Москва, 1965. ISBN 978-5-4467-1747-7

\bibitem{lovecraft} H. P. Lovecraft. \textit{Supernatural Horror in Literature.} The Recluse, 1927.

\end{thebibliography}

\end{document}